\section{Semana 2: Números Reais}
Agora nosso objeto de estudo é o conjunto dos números reais. Não será feita uma definição axiomática dos reais ( por meio de cortes de Dedekind e afins), nosso foco será provar que o conjunto dos números reais é um \textbf{corpo ordenado completo}. E mais: O conjunto dos reais é o único corpo ordenado completo.

\subsection{É um corpo}
Seja $\mathbb{K}$ um conjunto não vazio sobre o qual podemos definir duas aplicações binárias denominadas \textit{adição}(+) e \textit{multiplicação}(.); A estrutura $(\mathbb{K},+,.)$ é um corpo se: \begin{itemize}
    \item $(\mathbb{K},+)$ é um grupo abeliano;
    \item $(\mathbb{K}-\left\{0\right\},.)$ é grupo abeliano;
    \item A operação $.$ é distributiva em relação a $+$.
\end{itemize}

Ou seja, um corpo é dotado de todas as propriedades que gostamos e estamos acostumados. Dessa forma, é evidente que $(\mathbb{R},+,.)$, de fato, é um corpo. Outros exemplos de corpos são: $\mathbb{Q},\mathbb{C}$ e $\mathbb{Z}/p\mathbb{Z}$ com $p$ primo.

\begin{prob}
Prove que $x.0=0$ $\forall x \in \mathbb{R}$
\end{prob}
\begin{proof}
Temos que $x+0.x=1.x+0.x=x(1+0)=x.1=x \Rightarrow 0.x=0$
\end{proof}



\subsection{É um corpo ordenado}
Essa é a parte mais chata. Aqui vamos mostrar as relações de ordem no corpo dos reais, ou seja, falar sobre os conceitos de menor, maior e igual.
Dizer que o conjunto dos reais é um corpo ordenado significa dizer que existe um subconjunto $\mathbb{R}^{+} \subset \mathbb{R}$, denominado conjunto dos \textit{reais positivos}, que cumpre as condições:
\begin{itemize}
    \item A soma e o produto de reais positivos são positivos
    \item Dado $x \in \mathbb{R}$ ocorre exatamente uma das 3 possibilidades: ou $x=0$, ou $x \in \mathbb{R}^{+}$ ou $-x \in \mathbb{R}^{+}$
\end{itemize}
Definimos o conjunto $\mathbb{R}^{-}$ (\textit{reais negativos}) o conjunto dos números $x$ tal que $-x \in \mathbb{R}^{+}$. Dessa forma, concluimos que: $\mathbb{R}=\mathbb{R}^{+} \cup \mathbb{R}^{-}\cup\left\{0\right\}$.

Dizemos que $x$ é menor do que $y$ quando $y-x \in \mathbb{R}^{+}$, o conceito de maior é definido de forma análoga. É bom ressaltar que a relação de ordem $x<y$ possui algumas propriedades, entretanto não serão abordadas aqui, visto que já estamos habituados com todas elas.

$\mathbb{R},\mathbb{Q}(\sqrt{2})$ são exemplos de corpos ordenados.
\newpage
\begin{prob}
Se $x,y \in \mathbb{R}$ então $|x+y| \leq |x| + |y|$ 
\end{prob}
\begin{proof}
Como já conhecemos, a definição do valor absoluto de um número é a seguinte: $|x|=x$, se $x\geq0$ e $|x|$ $-x$ se $x<0$.

Para demonstrar a desigualdade, vamos usar o fato de que $|x|+|y| \geq x+y$ e também $|x|+|y| \geq -(x+y)$, logo $|x| + |y| \geq |x+y|=max\left \{x+y, -(x+y)  \right \}$
\end{proof}


\begin{theorem}
Sejam $a.x.\delta$ com $\delta>0$. Tem-se $|x-a| < \delta$ se, e só se, $x \in (a-\delta,a+\delta)$
\end{theorem}
\begin{proof}
De fato, $|x-a|<\delta \Leftrightarrow max\left \{ x-a,a-x \right \}<\delta \Leftrightarrow x-a < \delta$ e $a-x < \delta$, ou seja, $x<a+\delta$ e $x>a-\delta \Leftrightarrow x \in (a-\delta,a+\delta)$.
\end{proof}

\subsection{É um corpo ordenado completo}
Chegamos na parte mais legal. Até agora não conseguimos distinguir o corpo dos racionais do corpo dos reais. Essa semelhança será eliminada agora, pois o conjunto dos racionais não são um corpo ordenado completo.

Dizemos que um corpo ordenado é  \textbf{completo} se todo subconjunto não-vazio $A \subset \mathbb{R}$ \textit{limitado superiormente} possui um \textit{supremo}. Mas o que significa ser limitado superiormente e o que é o supremo? Um conjunto $A \subset \mathbb{R}$ é dito \textit{limitado superiormente} quando existe algum $b \in \mathbb{R}$ tal que $a \leq b$ $\forall a \in A$. Analogamente, definimos o conceito de \textit{limitado inferiormente}.

Seja $X \subset \mathbb{R}$ um conjunto não-vazio e limitado superiormente. Dizemos que b é o \textit{supremo} de $X$ quando $b$ é a menor cota superior de $X$, isto é, $b$ é o menor número que é maior que todos os elementos de $X$. Denotamos $b=supX$

Analogamente, definimos o \textit{ínfimo} de $X$ como sendo a maior cota inferior de $X$, isto é, o maior número que é menor que todos os elementos de $X$, seja $c$ tal número, denotamos $c=infX$

Vale a pena ressaltar que o supremo (e o ínfimo) de um conjunto nem sempre pertence ao intervalo o qual o conjunto está contido. Por exemplo, seja $X=[a,b)$, $a$ é o elemento mínimo, e portanto $a=infX$ e $b=supX$, veja porém, que $b \notin X $

Veja também que: Seja o conjunto $-X=\left \{ -x \mid x \in X \right \}$, então $infX=-(sup(-X))$, pois como $-X$ é limitado superiormente, então $X$ é limitado inferiormente.
\begin{theorem}[$\mathbb{R}$ é um corpo arquimediano] 
\begin{itemize}
    \item O conjunto $\mathbb{N} \subset \mathbb{R}$ não é limitado superiormente;
    \item O ínfimo do conjunto $X=\left \{ 1/n; n \in \mathbb{N} \right \}$ é 0;
    \item Dados $a,b \in \mathbb{R}^{+}$, $\exists n \in\mathbb{N} \mid n.a>b$
\end{itemize}
\end{theorem}
\begin{proof}
1) Se $\mathbb{N} \subset \mathbb{R}$ fosse limitado superiormente, existiria $c=sup\mathbb{N}$. Logo $c-1$ não seria uma cota superior de $\mathbb{N}$, isso é, existiria $n > c-1 \Rightarrow c<n+1$, logo $c$ não seria uma cota superior.\qedsymbol

2) Obviamente $0$ é uma cota inferior de $X$, basta provar então que não existe $c>0$ que é cota superior de $X$, isto é, queremos provar que $0$ é a maior cota inferior de $X$. Ora, dado $c>0$ existe, um natural $n>1/c \Leftrightarrow 1/n<c$. \qedsymbol

3) Dados $a,b \in \mathbb{R}^{+}$ por 1) sabemos que existe $n>b/a \Rightarrow na>b$.
\end{proof}

\begin{theorem}[Intervalos Encaixados]
Sejam $a_{n},b_{n} \in \mathbb{R}, n \in \mathbb{N}$ e $a_{1} \leq a_{2} \leq ...\leq... \leq b_{2} \leq b_{1}$,isto é, $[a_{1},b_{1}] \supset [a_{2},b_{2}] \supset ...$, ou de forma mais geral, $[a_{n},b_{n}] \supset [a_{n+1},b_{n+1}]$ $\forall n \in \mathbb{N}$, então $\bigcap_{n \in \mathbb{N}}[a_{n},b_{n}] \neq \varnothing $ 
\end{theorem}

\begin{proof}
Temos que o conjunto $A=\left \{a_{1},a_{2},...,a_{n},...  \right \}$ é limitado superiormente, e portanto, possui um supremo, seja $c=supA$, logo temos que $c \geq a_{n}$ e além disso, todo $c \leq b_{n}$ para todo $n \in \mathbb{N}$, portanto $c \in [a_{n},b_{n}]$ $\forall n \in \mathbb{N}$
\end{proof}

\begin{theorem}
O conjunto dos números reais é não enumerável e todo intervalo não degenerado é não enumerável.
\end{theorem}
\begin{proof}

\end{proof}

Um número é chamado de \textit{irracional} quando ele não é racional. Como o conjunto $\mathbb{Q}$ é enumerável, então pelo Teorema 2.4, de fato, existem números irracionais e mais: O conjunto dos irracionais ($\mathbb{I}=\mathbb{R}-\mathbb{Q}$) é não enumerável e portanto, formam a maioria dos reais. Ou seja, a maior parte do conjunto dos reais é formada por números irracionais.

\begin{theorem}[$\mathbb{Q}$ é denso em $\mathbb{R}$]
Todo intervalo não degenerado $I$ contém números racionais e irracionais
\end{theorem}

\begin{proof}
Com certeza, $I$ contém números irracionais, pois caso contrário seria enumerável. Para provar que $I$ possui números racionais, tomemos $[a,b] \subset I$, onde $a<b$ podem ser irracionais. Fixemos $n$ natural tal que $1/n < b-a$. Temos que $\mathbb{R}=\bigcup_{m\in \mathbb{Z}}^{}I_{m}$, onde $I_{m}=[m/n,(m+1)/n],m \in \mathbb{Z}$. Logo existe $m$ tal que $a \in I_{m}$. Como $a$ é irracional, $m/n < a < (m+1)/n$. Sendo $1/n$ o comprimento do intervalo $I_{m}$ menor que $b-a$, segue que $(m+1)/n < b$, logo o racional $(m+1)/n$ pertence ao intervalo $[a,b]$ e portanto, pertence a $I$.
\end{proof}

\begin{prob}
Prove que o conjunto dos polinômios com coeficientes inteiros é enumerável.
\end{prob}

\begin{proof}
Seja $\mathbb{Z}[x]$ o conjunto dos polinômios com coeficientes inteiros, sabemos que os elementos desse conjunto são da forma $P(x)=a_{n}x^{x}+...+a_{1}x+a_{0}$, com $a_{j} \in \mathbb{Z}$ que correspondem a uma lista $(a_{0},a_{1},...,a_{n})$, com $n+1$ elementos. Dessa forma seja $\mathcal{P}_n$ o conjunto dos polinômios com coeficientes inteiros de grau menor ou igual a $n$, conseguimos estabelecer uma sobrejeção $f:\mathbb{Z}^{n+1}\rightarrow \mathcal{P}_n$, logo $\mathcal{P}_n$ é enumerável. Como $\mathbb{Z}[x]=\bigcup_{n \in \mathbb{N}}\mathcal{P}_n$, isto é, $\mathbb{Z}[x]$ é uma reunião de conjunto enumeráveis, então ele é enumerável.
\end{proof}