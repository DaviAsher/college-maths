\section{Semana 1: Conjuntos Finitos e Infinitos}

Nessa primeira aula, é definido os conceitos de conjunto finito e infinito, também definido os conjuntos infinitos enumeráveis e não-enumeráveis.

\subsection{Números Naturais}
O conjunto dos números naturais $\mathbb{N}$ é definido pelos seguintes fatos, conhecidos como Axiomas de Peano:

\begin{theorem}
\begin{itemize}
    \item Existe uma função injetiva $s:\mathbb{N}\rightarrow \mathbb{N}$. A imagem $s(n)$ de cada número natural $n \in \mathbb{N}$ chama-se \textbf{sucessor} de $n$
    \item Existe um único natural $1 \in \mathbb{N}$ tal que 1 não é sucessor de nenhum outro natural.
    \item (Indução) Se um conjunto $X\subset \mathbb{N}$ é tal que $1 \in X$ e $n\in X\Rightarrow s(n)\in X$, então $X=\mathbb{N}$
\end{itemize}
\end{theorem}
Em posse dos axiomas de Peano, podemos definir a operação de \textit{adição} e de \textit{multiplicação}:
\begin{itemize}
    \item (Adição) $m+1=s(m)$ e $m+s(n)=s(m+n)$
    \item (Multiplicação) $m.1=m$ e $m.(n+1)=mn+m$
\end{itemize}

Embora não nos aprofundemos, vale a pena ressaltar as operações no conjunto dos naturais possuem todas as propriedades que já conhecemos na escola. A prova será omitida.


\begin{prob}
Prove que $n+1=1+n$ $\forall n \in \mathbb{N}$
\end{prob}

\begin{proof}
O caso $n=1$ é óbvio. Se vale para $n$, queremos mostrar que também vale para $s(n)$. Temos que: $$1+s(n)=s(n+1)\overset{hipotese}{=}s(1+n)=1+s(n)(*)$$
Agora, vamos provar que $n+m=m+n$ para todos $n,m \in \mathbb{N}$. Por indução, suponha $m+n=n+m$, daí veja que $$n+s(m)\overset{assoc.}{=}(n+m)+1=(m+n)+1\overset{assoc.}{=}m+(n+1)\overset{(*)}{=}(m+1)+n=s(m)+n$$
\end{proof}

\begin{prob}
Prove que para todo $n$ natural, não existe $p \in \mathbb{N}$ tal que $n < p < n+1$
\end{prob}
\begin{proof}
Suponha por absurdo que exista tal $p$. Então teremos $p=n+q$ e $n+1=p+r$. Daí $p+1=n+1+q=p+r+q\Rightarrow r+q=1$. E isso é um absurdo, pois somar 2 naturais é obter o sucessor de um número natural, que por definição, deve ser maior do que 1.
\end{proof}

\begin{theorem}
(Princípio da Boa Ordem): Todo subconjunto não vazio $A\subset \mathbb{N}$ possui um \textbf{único} menor elemento.
\end{theorem}

\begin{proof}
Se $1 \in A$ então 1 é o elemento mínimo. Definimos o conjunto $I_{n}$ como o conjunto dos números naturais menores ou iguais a $n$. Se $1\notin A$, então $A\cap I_{1}=\varnothing$. Seja $B=\left \{ n \in \mathbb{N} \mid A\cap I_{n}=\varnothing \right \}.$ Como A
não é vazio, então $B \neq \mathbb{N}$. Ou seja, $B$ não pode cumprir o 3º axioma de Peano. Como $1 \in B$, existe $n \in B$ tal que $s(n) \notin B$. Portanto $s(n)$ será o mínimo de $A$, pois ,de fato, $A\cap I_{n}=\varnothing$, mas $A\cap I_{n+1} \neq \varnothing $ e $ I_{n+1}= I_{n}\cup \left \{ n+1 \right \}$, logo $n+1 \in A$.
\end{proof}
Uma afirmação que é muito forte e útil é que o Princípio da Boa Ordem é \textit{equivalente} ao Princípio da Indução. A primeira parte da prova já foi feita a cima.

\begin{prob}
Prove o Princípio da Indução como uma consequência do Principio da Boa Ordem 
\end{prob}

\begin{proof}
Suponha $X \subset \mathbb{N}$ tal que $1 \in X$ e $n \in X\Rightarrow n+1 \in X$, obviamente 1 é o elemento mínimo de $X$. Dessa forma, se $X \neq \mathbb{N}$, o conjunto $\mathbb{N}-X$ possui um elemento mínimo $t$, de forma que $t=p+1$, logo $p \in X$, pois $p<k$. Portanto $p+1=k\notin X$, o que é uma contradição, visto que $1$ e $p$ pertencem a $X$. 
\end{proof}


\subsection{Conjuntos Finitos}
Dizemos que um conjunto $X$ é \textit{finito} quando é vazio ou quando existem $n \in \mathbb{N}$ e uma bijeção $f:X\rightarrow I_{n}$. Escrevendo $x_{j}=f(j)$ temos então $X=\left \{ x_{1},...,x_{n} \right \}$. A bijeção $f$ denomina-se \textit{contagem} dos elementos de $X$ e $n$ é o \textit{número de elementos} de $X$. Veja que $n$ é \textbf{único}.

\begin{theorem}
Se $A$ é um subconjunto próprio de $I_{n}$, não existe uma bijeção $f:A\rightarrow I_{n}$
\end{theorem}
\begin{proof}
Por absurdo, suponha que $n_{0}$ é o menor natural tal que existe uma bijeção $f:A \rightarrow I_{n_{0}}$.  Se $n_{0} \in A$, então existe uma bijeção $g:A \rightarrow I_{n_{0}}$ tal que $g(n_{0})=n_{0}$. Dessa forma a restrição de $g$ é uma bijeção de $A-\left \{ n_{0} \right \}$ sobre $I_{n_{0}-1}$, o que gera a contradição quanto a minimalidade do $n_{0}$. Caso $n_{0} \notin A$, então seja $a \in A$ com $f(a)=n_{0}$, teremos que a restrição de $f$ ao subconjunto próprio de $A-\left \{a \right \}$ será uma bijeção sobre $I_{n_{0}-1}$, o que contraria a minimalidade de $n_{0}$. 
\end{proof}

\begin{prob}
Prove o Princípio das Casas dos Pombos: Se $m>n$, para alojar $m$ pombos em $n$ casas é preciso que pelo menos uma casa abrigue mais de um pombo.
\end{prob}

\begin{proof}
O problema é equivalente a provar que não existe uma função injetiva $f: I_{m} \rightarrow I_{n}$. Se tal $f$ existe, então ela é uma bijeção do domínio de $I_{m}$ em $A=Im(I_{m})$ a qual está contida em $I_{n}$ e portanto é um subconjunto próprio de $I_{m}$, pois $m <n$, o que contraria o Teorema 1.3.
\end{proof}
\begin{theorem}
Todo subconjunto de um conjunto finito, é finito.
\end{theorem}

\begin{proof}
Se $X=\varnothing$, então o único subconjunto de $X$ é $\varnothing$, que é vazio.  (Completar em breve)
\end{proof}

\begin{corol}
Um subconjunto $X \subset \mathbb{N}$ é finito se, e só se, é limitado.
\end{corol}
\begin{proof}
Primeiramente, vamos definir o que é um subconjunto limitado: Um subconjunto $X \subset \mathbb{N}$ é \textit{limitado} quando existe $p$ natural tal que $x \leq p$ para todo $x \in X$. Dessa forma, se $X=\left \{ x_{1},x_{2},...,x_{n} \right \}$ é finito, pondo $p=\sum_{x_{j}\in X}^{}x_{j}$, vemos que $x\in X\Rightarrow x\leq p$. Reciprocamente, se $X \in \mathbb{N}$ é limitado, então $X \subset I_{p}$ para alguem $p \in \mathbb{N}$, logo pelo Teorema 1.4, $X$ é finito.
\end{proof}

\subsection{Conjuntos Infinitos}
Dizemos que um conjunto é \textit{infinito} quando ele não é finito (Sim, a definição é bem sem graça). Mais formalmente, $X$ é infinito quando, para qualquer $n \in \mathbb{N}$ não existe uma bijeção $f:X \rightarrow I_{n}$. O conjunto dos números naturais é infinito.

\begin{theorem}
Se $X$ é um conjunto infinito, existe uma função injetiva $f: \mathbb{N} \rightarrow X$.
\end{theorem}

\begin{proof}
Seja $A \subset X$ um subconjunto não vazio, escolhemos um elemento $x_{A} \in A$. Daí, pomos $f(1)=x_{X}$ e escrevemos $A_{n}=X-\left \{ f(1),...,f(n) \right \}$, veja que $A$ é nao vazio, pois $X$ é infinito. Então definimos $f(n+1)=x_{A_{n}}$. Para mostrar que é injetiva, sejam $n,m \in \mathbb{N}$. Suponha sem perda de generalidade $m < n$. Então $f(m) \in \left \{ f(1),...,f(n-1) \right \}$ enquanto $f(n) \in X-\left \{ f(1),...,f(n-1) \right \} $, logo $f(m) \neq f(n).$
\end{proof}

\begin{corol}
Um conjunto $X$ é infinito se, e só se, existe uma bijeção $\phi: X \rightarrow Y$, onde $Y$ é um subconjunto próprio de $X$.
\end{corol}

Desse corolário tiramos conclusões muito interessantes, que rondam nossa cabeça quando aprendemos sobre conjuntos infinitos com a tia Ana Maria na escola. Podemos concluir que há uma bijeção entre $\mathbb{N}$ e $\mathbb{N}_{k}$ onde $\mathbb{N}_{k}=\left \{k+1,k+2,... \right \}$. Entretanto uma observação mais legal é que \textit{há tantos números pares quanto números naturais}, essa relação é dada pela bijeção $\phi(n)=2n$. O mesmo vale para os ímpares, quando definimos a função $\psi(n)=2n-1$.

\subsection{Conjuntos Enumeráveis}
Agora será respondida outra pergunta que rodeia a cabeça do estudante do Ensino Fundamental quando aprende sobre números inteiros e racionais. O questionamento é: Qual conjunto tem mais elemento? O dos números inteiros ou o dos naturais? E os racionais?

Um conjunto $X$ é \textit{enumerável} quando é finito ou quando existe uma bijeção $f: \mathbb{N} \rightarrow X$, onde escrevemos $f(j)=x_{j}$, temos então $X=\left \{ x_{1},x_{2},...,x_{j},... \right \}$

\begin{theorem}
Todo subconjunto $X \subset \mathbb{N}$ é enumerável
\end{theorem}

\begin{proof}
Veja que, se $X$ é finito, não há o que demonstrar. Caso $X$ infinito, enumerados os elementos de $X$, colocando $x_{1}$ como o menor elemento de $X$ e definimos da mesma forma os demais $x_{j}$, escrevemos $A_{n}=X-\left \{ x_{1},x_{2},...,x_{n} \right \}$. Como $X$, por hipótese, é infinito, então $A_{n}$ não é vazio, dessa forma definimos $x_{n+1}$ como sendo o menor elemento de $A_{n}$, logo $X=\left \{ x_{1},x_{2},...,x_{n},... \right \}$. Caso existisse alguem $x \in X$ difererente de todos os $x_{n}$, teríamos $x \in A_{n}$ para todo $n$ natural, ou seja, $n$ seria maior que todos os elementos do conjunto infinito $X$, o que contraria a infinidade de $X$. 
\end{proof}

\begin{corol}
O produto cartesiano de dois conjuntos enumeráveis é enumerável
\end{corol}

\begin{proof}
O produto cartesiano de 2 conjuntos é definido da seguinte forma: $A\times B=\left \{ (a,b) \mid a \in A, b \in B \right \}$. Se $A$ e $B$ são enumeráveis, então existem as sobrejeções $f:\mathbb{N}\rightarrow A$ e $g:\mathbb{N}\rightarrow B$, portanto $\phi (a,b)=(f(a),g(b))$, onde $\phi: \mathbb{N}\times \mathbb{N}\rightarrow A\times B$, ( veja que $\phi$ é sobrejetiva).

Dessa forma, basta provar que $\mathbb{N}\times \mathbb{N}$ é enumerável. Para isso considere a função $\phi:\mathbb{N}\times \mathbb{N} \rightarrow \mathbb{N}$, dada por $\phi(a,b)=2^{a}.3^{b}$. Pelo Teorema Fundamental da Aritmética, $\phi$ é injetiva. Logo $\mathbb{N}\times \mathbb{N}$ é enumerável.
\end{proof}

\begin{corol}
A união de conjuntos enumeráveis é enumerável. Isto é, seja $X_{j}$ enumerável, temos que $X=\bigcup_{n=1}^{\infty }X_{n}$ é enumerável.
\end{corol}

Daí podemos concluir que o infinito enumerável "é o menor" dos infinitos. Ou seja, \textit{todo conjunto infinito contém um subconjunto infinito enumerável.}

\begin{prob}
Prove que o conjuntos dos números inteiros é enumerável
\end{prob}

\begin{proof}
Seja $\mathbb{Z}=\left \{ ...,-2,-1,0,1,2,... \right \}$ o conjunto dos números inteiros. Definimos a bijeção $f:\mathbb{N}\rightarrow \mathbb{Z}$, onde \begin{itemize}
    \item $f(n)=\frac{(n-1)}{2}$ se $n$ é ímpar;
    \item $f(n)=-\frac{n}{2}$ se $n$ é par.
\end{itemize}
Logo veja que há tantos números negativos quanto pares e tantos positivos quanto ímpares.
\end{proof}

\begin{prob}
Prove que o conjunto dos racionais é enumerável.
\end{prob}
\begin{proof}
Temos que $\mathbb{Q}=\left \{ \frac{m}{n} \mid m,n \in \mathbb{Z}, n \neq 0 \right \}$. Já sabendo que o conjuntos dos números inteiros é enumerável, definimos uma função sobrejetora $f:\mathbb{Z}\times \mathbb{Z}^{\ast }\rightarrow \mathbb{Q}$, tal que, $(m,n)\mapsto \frac{m}{n}$.
\end{proof}