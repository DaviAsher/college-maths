\section{Séries Numéricas}
\subsection{Definições}
Dada uma sequência $(a_{n})_{n \in \mathbb{N}}$ de reais, formamos uma nova sequência $(s_{n})_{n \in \mathbb{N}}$, onde $s_{n}=\sum_{j=1}^{n}a_{j}$. Dessa forma, definimos uma \textit{série }como $s=\sum a_{n}=\ \lim_{n\to\infty} (a_{1}+...+a_{n}) \ $, onde os $s_{j}$ são as somas parciais da série. 

Dizemos que $s=\sum_{n=1}^{\infty }a_{n}$ converge para $L$ se $\lim_{n\to\infty} (s_{n})=L$. Caso esse limite não exista, dizemos que a série é \textit{divergente.}

\subsection{Alguns exemplos e resultados}

\begin{exem}(\textbf{Série Geométrica})
Para $|a|<1$, a série $s=\sum_{n=0}^{\infty }a^{n}$ é convergente, com $s=\frac{1}{1-a}$. Também temos que $s=\sum_{n=0}^{\infty}q^{n}a^{n}$ converge, sendo $s=\frac{1}{1-qx}$
\end{exem}

\begin{exem}
A série $s=\sum \frac{1}{n(n+1)}$ converge. Temos que o termo geral da sequência é $a_{n}=\frac{1}{n(n+1)}=\frac{1}{n}-\frac{1}{n+1}$, de forma que a soma parcial $s_{n}=1-\frac{1}{n+1}$. De fato, quando $n$ fica muito grande, $\frac{1}{n+1}$ se aproxima, a medida que o $n$ cresce, do 0. Dessa maneira, $lims_{n}=1$, logo,$s=\sum \frac{1}{n(n+1)}=1$
\end{exem}

\begin{prob}(\textbf{Série Harmônica})
Prove que, a série $s=\sum_{n=1}^{\infty} \frac{1}{n}=+\infty$, ou seja, é divergente.
\end{prob}

\begin{proof}
Vamos fazer uma estimativa, efetuando a soma até uma potência de 2, dessa forma temos: $s=1+\frac{1}{2}+\frac{1}{3}+...+\frac{1}{2^{n}}$, de fato, esse valor vai ser maior do que quando trocamos $\frac{1}{3}$ por $\frac{1}{4}$, $\frac{1}{5},\frac{1}{6},\frac{1}{7}$ por $\frac{1}{8}$ e assim por diante de modo a obter $s=1+\frac{1}{2}+\frac{1}{3}+...+\frac{1}{2^{n}} < 1+\frac{1}{2}+\frac{1}{4}+\frac{1}{4}+\frac{1}{8}+...+\frac{1}{2^{n}}+\frac{1}{2^{n}}$, de modo que a potência $2^{k}$ no denominador de   $2^{k-1}$ parcelas. Portanto $1+\frac{1}{2}+\frac{1}{4}+\frac{1}{4}+\frac{1}{8}+...+\frac{1}{2^{n}}+\frac{1}{2^{n}}=1+\frac{1}{2}+\frac{2}{4}+...+\frac{2^{n-1}}{2^{n}}=1+\frac{n}{2}\rightarrow \infty$.
\end{proof}

\begin{theorem}(\textbf{Critério de Comparação})
Sejam $\sum a_{n}$ e $\sum b_{n}$ séries de termos positivos. Suponha que exista $n_{0} \in \mathbb{N}$ e $c>0$ tais que $n>n_{0}\Rightarrow a_{n} \leq c.b_{n}$ para todo $n>n_{0}$. Então se $\sum b_{n}$ converge, então $\sum a_{n}$ converge e se $\sum a_{n}$ diverge, então $\sum b_{n}$ diverge.
\end{theorem}

\begin{proof}
Sejam $s_{n}=\sum_{k=1}^{n}a_{k}$ e $t_{n}=\sum_{k=1}^{n}b_{k}$, temos que as sequências $(s_{n})$ e $(t_{n})$ são não-descrescentes. Temos que para todo $n \in \mathbb{N}$, $s_{n}\leq s_{n_{0}} + c.t_{n}$. Logo se $L=lim t_{n}=sup t_{n}$ daí chegamos em $s_{n}\leq s_{n_{0}} +c.L$
\end{proof}

\begin{theorem}
Se $\sum a_{n}$ é uma série convergente, então $lim a_{n}=0$
\end{theorem}
\begin{proof}
Temos que $a_{n}=s_{n}-s_{n-1}\Rightarrow lim a_{n}=0 = lim s_{n}-lim s_{n-1}$, como a sequência converge, temos que $lim s_{n}=S=lims_{n-1}$, logo $lima_{n}=S-S=0$.
Note que a recíproca não é verdade, o limite do termo geral da série pode ser 0, mas isso não implica que ela será convergente; como, por exemplo, na série harmônica.
\end{proof}
\newpage
\subsection{Séries absolutamente convergentes}
Dizemos que uma série $\sum a_{n}$ é \textit{absolutamente convergente} quando $\sum |a_{n}|$ converge. Uma série convergente cujos termos não mudam de sinal, por exemplo, é uma série absolutamente convergente.

\begin{theorem}
Se uma série converge absolutamente, então ela é convergente
\end{theorem}

\begin{proof}
Seja $s_{n}=\sum_{k=1}^{n}a_{k}$ e $t_{n}=\sum_{k=1}^{n}|a_{k|}$, temos que $t_{n}$ converge, como pelo problema 3.2, temos que $t_{n}$ é de Cauchy. Portanto $\forall \varepsilon>0,\exists n_{0} \in \mathbb{N}$ tal que $n_{0} \leq m < n \Rightarrow |t_{n}-t_{m}|=\sum _{k=m+1}^{n}|a_{k}|<\varepsilon \Rightarrow |s_{n}-s_{m}=|\sum_{k=m+1}^{n}a_{k}| \leq \sum _{k=m+1}^{n}|a_{k}|<\varepsilon$, logo $s_{n}$ é de Cauchy e portanto, converge.
\end{proof}

Logo, podemos dizer, pelo Teorema acima, que se tomarmos uma série convergente cujos os termos são todos não negativos, se trocarmos de maneira aleatória o sinal de alguns termos, mesmo que seja de um número infinito de termos, a sequência continuará convergindo.

Se uma série converge, mas não converge absolutamente, dizemos que ela é \textit{condicionalmente convergente}

\begin{theorem}(\textbf{Leibniz})
Se $(a_{n})$ é uma sequência decrescente com $lim a_{n}=0$, então $\sum (-1)^{n+1}a_{n}=a_{1}-a_{2}+a_{3}-a_{4}+...$ é uma série convergente.
\end{theorem}
\begin{proof}
Comece observando que a sequência tender a 0 e ser decrescente implica necessariamente que todos os termos da sequência são não-negativos. Seja $s_{n}=\sum_{k=1}^{n}(-1)^{k+1}a_{k}$. Temos $s_{2} \leq s_{4} \leq ... \leq s_{2n}\leq...\leq s_{2n-1}\leq...\leq s_{3} \leq s_{1}$. De fato, como $s_{n}=a_{1}-a_{2}+...+(-1)^{n+1}a+{n}$, então $s_{2n}=s_{2n-2}+a_{2n-1}-a_{2n}$ e $s_{2n+1}=s_{2n-1}-a_{2n}+a_{2n+1}$. Portanto, as reduzidas de ordem par formas uma sequência não decrescente e as de ordem ímpar formam uma sequência não crescente, o que demonstra a desigualdade e $lim s_{2n}=lim s_{2n-1}$, visto que $lim a_{n}=0$.
\end{proof}

\begin{exem}
Pelo teorema 5.2 as séries $\sum(-1)^{n+1}log(1+\frac{1}{n})$ e $\sum \frac{(-1)^{n+1}}{n}$ são (condicionalmente) convergentes.
\end{exem}

\subsection{Testes de convergência}
Não há nenhum teste simples que estabeleça uma condição necessária e suficiente para que uma séria seja convergente, entretanto há alguns testes que "quebram um galho".

Um resultado preliminar importante é: Seja uma série absolutamente convergente $\sum b_{n}$ com $b_{n} \neq0$. Se a sequência $(\frac{a_{n}}{b_{n}})$ for convergente, então a série $\sum a_{n}$ será absolutamente convergente. A demonstração desse resultado segue imediatamente pelo Teorema 4.1.

\begin{theorem}(\textbf{Teste de d'Alembert})
Seja $a_{n}\neq 0, \forall n \in \mathbb{N}$. Se $\exists c \in (0,1)$ tal que $|\frac{a_{n+1}}{a_{n}}|\leq c<1$ para todo $n$ suficientemente grande(em particular se $lim|\frac{a_{n+1}}{a_{n}}|<1$) então a série $\sum a_{n}$ é absolutamente convergente.
\end{theorem}

\begin{proof}
De fato, $\exists n_{0} \in \mathbb{N}$ tal que $|\frac{a_{n+1}}{a_{n}}|\leq c <1$. Segue por indução que $|a_{n}|\leq |a_{n_{0}}|.c^{n-n_{0}}\Rightarrow \frac{|a_{n}|}{c^{n}}\leq \frac{|a_{n_{0}}|}{c^{n_{0}}}$, donde $\frac{|a_{n}|}{c^{n}}$ é limitada e como $\sum c^{n}$ é absolutamente convergente (pois $|c|<1$), então a série $\sum a_{n}$ também o é.
\end{proof}

Ao aplicar o teste de d'Alembert procuramos calcular o limite $|\frac{a_{n+1}}{a_{n}}|=L>0$, se $L>1$ então obviamente a série diverge, visto que a sequência será estritamente crescente. Caso ocorra $L=1$ o teste é inconclusivo.

\begin{exem}
A partir do teste de d'Alembert sabemos que as séries $\sum \frac{a^{n}}{n!};\sum \frac{n!}{n^{n}};\sum \frac{n^{k}}{a^{n}}$ são convergentes para $a>1$. Daí chegamos ao fato de que, para $n$ grande, $n^{k}\ll a^{n} \ll n! \ll  n^{n}$.
\end{exem}

\begin{theorem}(\textbf{Teste de Cauchy})
Se existe $c \in (0,1)$ tal que $\sqrt[n]{|a_{n}|}\leq c<1$ (em particular, quando $lim\sqrt[n]{|a_{n}|}<1$), então a série $\sum a_{n}$ converge absolutamente.
\end{theorem}

\begin{proof}
Se $\sqrt[n]{|a_{n}|}\leq c<1$ então $|a_{n}|\leq c^{n}\Rightarrow |a_{n}|\leq c^{n}$ é limitada, então $\sum a_{n}$ é convergente.
\end{proof}
Da mesma forma que ocorre com o teste de d'Alembert, seja $lim \sqrt[n]{a_{n}}=L$, se $L>1$ a sequência diverge e se $L=1$ o teste é inconclusivo.

\begin{exem}
Seja $a_{n}=(\frac{log n}{n})^{n}$. Como $\sqrt[n]{a_{n}}=\frac{log n}{n}$ tende a zero, então $\sum a_{n}$ é convergente.
\end{exem}

\begin{theorem}\footnote{Este teorema relaciona o teste de Cauchy (ou da raiz) e o teste de d'Alembert (ou da razão). De forma que o teste da razão implica o teste da raiz. }
\begin{itemize}
    \item Se $\exists c<1$ tal que $|\frac{a_{n+1}}{a_{n}}|\leq c$ para todo $n$ grande, então existe $c'$ tal que $\sqrt[n]{|a_{n}|}< c'$
    \item Se $lim |\frac{a_{n+1}}{a_{n}}|=L$, então $\sqrt[n]{a_{n}}=L$
\end{itemize}
\end{theorem}


